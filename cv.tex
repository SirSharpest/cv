% Local Variables:
% TeX-engine: xetex
% End:
\documentclass[a4paper,10pt]{article}

% A Few Useful Packages
\usepackage[margin=0.5in]{geometry}
\usepackage{longtable}
\usepackage{marvosym}
\usepackage{fontspec}                                   %for loading fonts
\usepackage{xunicode,xltxtra,url,parskip}       %other packages for formatting
\RequirePackage{color,graphicx}
\usepackage[usenames,dvipsnames]{xcolor}
\usepackage[big]{layaureo}                              %better formatting of the A4 page
% an alternative to Layaureo can be ** \usepackage{fullpage} **
\usepackage{supertabular}                               %for Grades
\usepackage{titlesec}					%custom \section
\usepackage{multirow}
\usepackage{bibentry}
\makeatletter\let\saved@bibitem\@bibitem\makeatother
\usepackage{natbib}


\tolerance=1
\emergencystretch=\maxdimen
\hyphenpenalty=10000
\hbadness=10000
% Setup hyperref package, and colours for links
%\usepackage{hyperref}
%\definecolor{linkcolour}{rgb}{0,0.2,0.6}
%\hypersetup{colorlinks,breaklinks,urlcolor=linkcolour, linkcolor=linkcolour}

\usepackage[colorlinks = true,
            linkcolor = blue,
            urlcolor  = blue,
            citecolor = blue,
            anchorcolor = blue]{hyperref}
\makeatletter\let\@bibitem\saved@bibitem\makeatother

\addtolength{\oddsidemargin}{-.575in}




% CV Sections inspired by:
% http://stefano.italians.nl/archives/26
% Tweaked by Nathan Hughes
% (@sirsharpest)
\titleformat{\section}{\Large\scshape\raggedright}{}{0em}{}[\titlerule]
\titlespacing{\section}{0pt}{3pt}{3pt}
% Tweak a bit the top margin
% \addtolength{\voffset}{-1.3cm}


\hyphenation{im-pre-se}

% -------------WATERMARK TEST [**not part of a CV**]---------------
\usepackage[absolute]{textpos}

\setlength{\TPHorizModule}{30mm}
\setlength{\TPVertModule}{\TPHorizModule}
\textblockorigin{2mm}{0.65\paperheight}
\setlength{\parindent}{0pt}

% --------------------BEGIN DOCUMENT----------------------
\begin{document}
\nobibliography{library.bib}
\bibliographystyle{mystyle}
\pagestyle{empty} % non-numbered pages

\font\fb=''[cmr10]'' %for use with \LaTeX command

% --------------------TITLE-------------

\par{\centering
  \vspace{-2ex}
  {\Huge Nathan {Hughes}
  }\bigskip\par}

% --------------------SECTIONS-----------------------------------
% Section: Personal Data

\begin{center}
  \begin{tabular}{rl}
    \textsc{Nationality} & Irish/British  \\
    \textsc{Date of Birth:} & 05                May 1994 \\
    \textsc{email:}     & \href{mailto:Nathan.Hughes@jic.ac.uk}{Nathan.Hughes@jic.ac.uk} | \href{mailto:nathan1hughes@gmail.com}{nathan1hughes@gmail.com} \\
    \textsc{website:}  & \href{sirsharpest.github.io}{sirsharpest.github.io}
  \end{tabular}
\end{center}


\section{Personal Statement}
I am an extremely driven and motivated PhD student. With a true thirst for knowledge and desire for understanding, it is my long-term goal to pursue a career in academia.

\section{Employment / Voluntary History}
  \begin{longtable}{r|p{11.5cm}}
    %%%


        &Journal of Open Source Software - Reviewer  \\\textsc{August 2019 - Ongoing}
    &\footnotesize{JOSS is an open source journal which I regularly review for}\\\multicolumn{2}{c}{} \\

    &University of East Anglia - Science outreach  \\\textsc{May 2019 - Ongoing}
    &\footnotesize{Working with the science outreach team at UEA to bring accessible science lessons to all. This is a part-time role}\\\multicolumn{2}{c}{} \\

    &Doonan Lab - Bioinformatician \\\textsc{Sept 2017 - Sept 2018}
    &\footnotesize{Here I analysed and prepared large data sets for publication in the area of crop-research.}\\\multicolumn{2}{c}{} \\

    &National Plant Phenomics Centre -  Systems Developer \\\textsc{May 2016 - Aug 2017}
    &\footnotesize{My role at the NPPC was extremely varied. I worked on building and designing a Gravimetrics system for plant phenotyping and data generation, writing image analysis software. A large part of my time was devoted to data analysis and statistical evaluation of data.}\\\multicolumn{2}{c}{} \\

    %%%%

    &Aberystwyth University -  Demonstrator \\ \textsc{Sept 2015 - June 2018}
    &\footnotesize{I was a mentor to first and second year students during their workshops. Specifically I helped to reinforce the knowledge students learn in their lectures.}\\\multicolumn{2}{c}{} \\
    %%%%

    %%%%
    &Belfast Metropolitan College - Systems Specialist \\\textsc{May - Sept 2013}
    &\footnotesize{I had been employed as an IS advisor/specialist by BMC to implement their multi-campus systems upgrade.}\\\multicolumn{2}{c}{} \\

    &Salto National Gymnastics Centre - Gymnastics Coach \\\textsc{Feb 09 - March 2013}
     & \footnotesize{Sports coach for national level athletes}\\\multicolumn{2}{c}{} \\
  \end{longtable}

% Section: Education
\section{Education}
\begin{tabular}{llllr}

  \textsc{Current}  & \textbf{PhD} & Computation Biology & John Innes Centre &   \\

  \textsc{June} 2018  & \textbf{BSc} & Computer Science & Aberystwyth University & \textbf{1:1} \\

  \textsc{May} 2014& \textbf{Diploma} & Software Engineering & Belfast Metropolitan College &  \textbf{Distinction}

\end{tabular}


\section{Publications}

\begin{itemize}
\item{\bibentry{hughesNonDestructiveHighContentAnalysis2017}}
\item{\bibentry{hughesAnalysisWheatGrain}}
\item{\bibentry{cook2018barley}}
\item{\bibentry{unpublishedMe}}
\end{itemize}



\section{Research Experience}

My current project is focused on uncovering the mechanisms in which cell-to-cell communication operates, specifically in examining \textit{plasmodesmata} and how molecules move from one cell to another.

Previously my work has been primarily focused on genomic and phenotypic research in crop science. Specifically computer vision analysis and QTL mapping of seed and grain traits. Additionally I have spent considerable time designing and building automated systems for gathering and generating scientific data.


\section{Skills and Knowledge}


\begin{tabular}{p{3cm}l}


  Research skills: & Statistical analysis, Research methods, Image analysis, Academic writing

  \\

  \\

  Practical skills: & Hardware design, Pneumatic construction, PCB design, \\& Electrical engineering,  Product invention

    \\

  \\

  Experience: & Machine learning, Bayesian methods, Data mining, \\& Artificial intelligence, Data modelling

  \\

  \\


  Programming languages: & C/C++, Python, R, Java, MATLAB, BASH, \LaTeX,  Haskell, \vspace{-0.5cm} \\ &  Ruby, HTML, PHP, Perl, Lisp


\end{tabular}


\section{Grants / Awards}
\begin{tabular}{ll}
  \textsc{UK Open Source Awards - Student Category 2019}& \textbf{1st place}
  \\
  \textsc{Genetics Society Research Grant 2017}& \textbf{Recipient}
  \\
  \textsc{UK-RAS Field Robotics Contest 2016}& \textbf{3rd place}
  \\
  \textsc{Aberystwyth Excellence Scholarship}& \textbf{Recipient}
\end{tabular}


\section{Research Talks / Lectures}
\begin{tabular}{p{8cm}ll}
  \textsc{Genetics Society} & \textbf{Research presentation} & 2017  \\
  \textsc{FOSDEM} & \textbf{Lightning Talk} & 2016  \\
  \textsc{British Computer Society} & \textbf{Project update} & 2018 \\
  \textsc{Aberystywth University} & \textbf{Bioinformatics Conference} & 2017

\end{tabular}


\section{Open source projects}


\begin{tabular}{r|p{10cm}}

  Grain analysis software  & To streamline a data analysis pipeline, I developed a set of tools to allow for straight-forward analysis of 3D grain parameters (\href{https://github.com/SirSharpest/Grain\_Analyser\_GUI}{https://github.com/SirSharpest/Grain\_Analyser\_GUI})\\
  \\

  Wikipedia contributions & Science is most useful when it is accessible, in my spare time I contribute and write articles on \href{Wikipedia.org}{Wikipedia.org}

  \\
  Biology of the cell course & I started a small project to make an openly available crash-course for anyone from a non-biology background. (\href{https://github.com/SirSharpest/Bio-Cramming}{https://github.com/SirSharpest/Bio-Cramming})\\
  \\
  Image analysis software& For fun, I built a  Rubik's cube solver which uses OpenCV (\href{https://github.com/SirSharpest/Rubkis-Cube-Solver}{https://github.com/SirSharpest/Rubkis-Cube-Solver})
\end{tabular}

\section{References}
\begin{tabular}{lll}
  \textsc{Prof R. Morris} & \textbf{Plant Health Programme Leader, JIC
}  & \href{Richard.Morris@jic.ac.uk}{Richard.Morris@jic.ac.uk}
  \\
  \textsc{Prof J. Doonan} & \textbf{Director, NPPC}  & \href{jhd2@aber.ac.uk}{jhd2@aber.ac.uk}
  \\
  \textsc{Dr H. Dee} & \textbf{Senior Lecturer, Aberystwyth University}  & \href{hmd@aber.ac.uk}{hmd@aber.ac.uk}
  \\
  \textsc{Dr C. Sauze} & \textbf{Data Manager NPPC}  & \href{cos@aber.ac.uk}{cos@aber.ac.uk}
\\
  \textsc{Mr P. Greenwood} & \textbf{Project Manager, BMC}  & \href{PGreenwood@belfastmet.ac.uk}{PGreenwood@belfastmet.ac.uk}
  \\
\end{tabular}
\end{document}
% Local Variables:
% TeX-engine: xetex
% TeX-master: t
% End:
